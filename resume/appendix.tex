%-------------------------------------------------------------------------------
% SECTION TITLE
%-------------------------------------------------------------------------------
\cvsection{Appendix}

\vspace{-2mm}

%-------------------------------------------------------------------------------
% CONTENT
%-------------------------------------------------------------------------------
\begin{cventries}

%---------------------------------------------------------
  
\cventry
  {} % Degree
  {Research intern: Cuju} % Institution
  {} % Location
  {https://github.com/Cuju-ft/Cuju} % Date(s)
  {
    \vspace{-2mm}
    \begin{cvitems} % Description(s) bullet points
      \item {Frequent VM state backup required in system migration led to high performance overhead and RTD(Round-Trip Delay) by about 40 times.}
      \item {Cuju reduced the processing overhead of fault tolerance with optimizations like dirty state tracking, boosting performance using fake ACK to unlock traffic rate control within VMs, reducing latency by 97\%}
      \item {Connected VMs with virtual bridge networks on tap interfaces and also with the host machine with NIC configuration.}
      \item {Troubleshot integrity failures using official tests like \emph{qtest}, \emph{iotest}, and \emph{kvm tests}, where failures include functionality not implemented and invalid function call paths. }
      \item {Led to further master study on networking at Tsinghua University.}
    \end{cvitems}
  }

  \vspace{-3mm}

  \cventry
    {} % Degree
    {ATP: In-network Aggregation for Multi-tenant Learning.} % Institution
    {} % Location
    {https://www.usenix.org/conference/nsdi21/presentation/lao} % Date(s)
    {
      \vspace{-2mm}
      \begin{cvitems} % Description(s) bullet points
        \item {Incast traffic from workers to parameter servers led to scalability bottlenecks in distributed machine learning, impacting even more on communication-heavy models like VGGs.}
        \item {With a coherent design between network switches and end servers, ATP accelerates the training process by up to 38\% - 66\%.}
        \item {Built custom traffic rate and congestion control in a multi-tenant scenario with variable background traffic patterns.}
        \item {Optimizations including converting floating-point operations to fixed-pointed integers due to the lack of P4 semantic support and packet segmentation due to restricted header parsing ability.}
        \item {Integrated ATP with different tools like PyTorch, TensorFlow ,and MXNet.}
      \end{cvitems}
    }

    \vspace{-3mm}
  
    \cventry
      {} % Degree
      {NQ/ATP: Massive Aggregate Queries in Data Center Networks.} % Institution
      {} % Location
      {https://ieeexplore.ieee.org/document/9812906/} % Date(s)
      {
        \vspace{-2mm}
        \begin{cvitems} % Description(s) bullet points
          \item {Extensive research based on ATP, making in-network aggregation adapted to the nature of dynamic routing in scalable data-center networks.}
          \item {Existing solutions with in-network computation failed under a non-deterministic routing path.}
          \item {Binding the routing protocol along with the multi-layered aggregation, NQ/ATP resolved the congested traffic by 87.5\% and saves up at most 97.6\% of switch memory usage.}
          \item {Based on a revised version of source routing, NQ/ATP constructs the necessary routing fields on-the-fly, where routing headers are shared by all logically-equivalent switches in dynamic routing.}
        \end{cvitems}
      }

      \vspace{-3mm}
  
      \cventry
        {} % Degree
        {Collaborative project with Hwawei: Gateway offloading} % Institution
        {} % Location
        {} % Date(s)
        {
          \vspace{-2mm}
          \begin{cvitems} % Description(s) bullet points
            \item {Software gateways on commodity servers came with large capacity but low throughput. Thus, we offloaded the elephant flows to network switches with high throughputs.}
            \item {Implemented a coherent gateway between network switches and servers, servers predicting flow patterns and located elephant flows with the top-K "predicted" traffic.}
            \item {Proposed a distributed gateway design to achieve higher bandwidth, splitting the flow space evenly to each software gateway based on the hashed result, evaluated with its fairness index.}
          \end{cvitems}
        }

        \vspace{-3mm}
  
        \cventry
          {} % Degree
          {NFD2Rust: Interpret NFD with Rust} % Institution
          {} % Location
          {https://github.com/chenIshi/NFD2Rust} % Date(s)
          {
            \vspace{-2mm}
            \begin{cvitems} % Description(s) bullet points
              \item {Interpret NFD, a Cpp-like language to describe network virtual functions, in addition to describing network terms like TCP SYN/ACK fields when parsing.}
              \item {As the first project in my master;s degree, the Rust language, and network virtualization functions, I actively cooperate with my senior colleague, the inventor of NFD, to clarify the syntax and behavior of NFD.}
            \end{cvitems}
          }

        \vspace{-3mm}
  
        \cventry
          {} % Degree
          {Demystifying the Linux CPU Scheduler} % Institution
          {} % Location
          {https://github.com/sysprog21/linux-kernel-scheduler-internals} % Date(s)
          {
            \vspace{-2mm}
            \begin{cvitems} % Description(s) bullet points
              \item {Elaborate on the Linux scheduler design iteration, each with hands-on experiments to identify the physical behaviors within the Linux kernel using profiling tools.}
            \end{cvitems}
          }

%---------------------------------------------------------
\end{cventries}
